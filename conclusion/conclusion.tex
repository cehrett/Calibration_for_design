\documentclass[10pt,a4paper]{article}
\usepackage[utf8]{inputenc}
\usepackage[T1]{fontenc}
\usepackage{amsmath}
\usepackage{amsfonts}
\usepackage{amssymb}
\usepackage{graphicx}
\usepackage{natbib}
\author{Carl Ehrett}
\title{Conclusion}



\begin{document}
	
\maketitle

\section{Benefits}

The research presented in the previous chapters addresses two distinct desiderata related to model-assisted design.
Firstly, there is the desideratum of undertaking model-assisted design in a way that accounts for all forms of uncertainty -- uncertainty due to the model inputs, uncertainty due to the stochastic nature of the objective function, and/or uncertainty due to observation error of the outputs.
All of these sources of uncertainty can be modeled and included in the Bayesian framework used to employ our methodology.
The resulting posterior distribution of the design inputs quantifies uncertainty as to what inputs could lead to optimal system behavior.
The corresponding posterior predictive distributions quantifies uncertainty as to that resulting system behavior, including uncertainty of the entire Pareto front of the system.
In contrast with approaches such as that of \citet{Olalotiti-Lawal2015}, who provide similar uncertainty quantification of design input settings using a distribution they contrive, our method provides the uncertainty in a posterior distribution that is directly dictated by what is known about the computer model itself, by one's prior knowledge about the appropriate design settings, and by the priorities of decision-makers.

Our method furthermore evades the need to be able to evaluate the objective function adaptively.
This requirement is a limitation shared by most other Bayesian optimization (BO) methods.
In this way, our method may be employed in scenarios where researchers are confined to the usage of pre-existing data sets, or in scenarios where the experimental design used for data-gathering must satisfy priorities other than that of engineering design.

Secondly, there is the desideratum of unifying procedures for calibration and design.
Typically these two tasks are undertaken separately; a model would be calibrated and then the calibrated model would be put to use for model-assisted design.
However, design priorities arise for models that stand in need of calibration, and wedding the frameworks for calibration and design allows for a single use of a dataset to satisfy both sets of goals.



\bibliographystyle{apalike}

\bibliography{lit_review}
	
\end{document}